\subsection{Method}
To research a new form of calendar I will be utilizing case studies with a small specially selected individuals.
I will do this because when you are trying to rethinking and explaining a metaphor that is deeply ingrained as our calendar will take lots of One on One time to get across.
To get the idea across correctly is the first step the second step is to get the participants thinking and working within such a mindset that all of their events are not static and they don't have to be planned in such a manor.
It will be the shift for scheduling to planning and executing on those plans in an intelligent manor. Now one could say that an AI can follow through with these plans better than we can so way worry about trying to get the end user in the mess of pruning and fine-tuning how they work through their everyday problems.
It's hard for me not to see the value in the latter, but I will elucidate.
The reason for having a user go through the algorithmic steps to solving their day is the same as having a toddler go through the trouble of learning the written language.
Yes we have device now that can speak the words on the screen and even capture our spoken tongue, but you and I both know that it would be awfully hard to obtain those benefits with knowing the written word in the first place.
What I am trying to say is that you don't need something when you know it but if you don't know it then its exponentially hard to have the same results even with the best of the tooling allowed.
On that note the reason for doing a case study will be to make sure that this training will groom the individuals trying out the app and will make it a part of the way they think.
I wont be able to get the same kind of feedback from a large sample size.

\subsection{Tooling}
I will be using an iPhone Application for this. I choose the iPhone because its has a suite of development API's that can really tie the vision together in a quick enough time frame for this research project. The application will be a a list view calendar that the user can enter in events and don't have a predefined end date. But with the data that will be provided in the calendar the user will be able to mark off what they were doing so much more efficiently and accurately than before.

